\documentclass[twoside]{article}

\usepackage[sc]{mathpazo} % Use the Palatino font
\usepackage[T1]{fontenc} % Use 8-bit encoding that has 256 glyphs
\linespread{1.05} % Line spacing - Palatino needs more space between lines
\usepackage{microtype} % Slightly tweak font spacing for aesthetics

\usepackage[hmarginratio=1:1,top=32mm,columnsep=20pt]{geometry} % Document margins
\usepackage{multicol} % Used for the two-column layout of the document
\usepackage[hang, small,labelfont=bf,up,textfont=it,up]{caption} % Custom captions under/above floats in tables or figures
\usepackage{booktabs} % Horizontal rules in tables
\usepackage{float} % Required for tables and figures in the multi-column environment - they need to be placed in specific locations with the [H] (e.g. \begin{table}[H])
\usepackage{hyperref} % For hyperlinks in the PDF

\usepackage{lettrine} % The lettrine is the first enlarged letter at the beginning of the text
\usepackage{paralist} % Used for the compactitem environment which makes bullet points with less space between them

\usepackage{abstract} % Allows abstract customization
\renewcommand{\abstractname}{Summary} % Change "abstract" to "Summary"
\renewcommand{\abstractnamefont}{\normalfont\bfseries} % Set the "Abstract" text to bold
\renewcommand{\abstracttextfont}{\normalfont\small\itshape} % Set the abstract itself to small italic text

\usepackage{titlesec} % Allows customization of titles
\renewcommand\thesection{\Roman{section}} % Roman numerals for the sections
\renewcommand\thesubsection{\Roman{subsection}} % Roman numerals for subsections
\titleformat{\section}[block]{\large\scshape\centering}{\thesection.}{1em}{} % Change the look of the section titles
\titleformat{\subsection}[block]{\large}{\thesubsection.}{1em}{} % Change the look of the section titles

\usepackage{fancyhdr} % Headers and footers
\pagestyle{fancy} % All pages have headers and footers
\fancyhead{} % Blank out the default header
\fancyfoot{} % Blank out the default footer
\fancyhead[C]{Project Proposal $\bullet$ April 2015} % Custom header text
\fancyfoot[RO,LE]{\thepage} % Custom footer text

%-----------------------------------------------------------------------------
%	TITLE SECTION
%-----------------------------------------------------------------------------

\title{\vspace{-15mm}\fontsize{20pt}{10pt}\selectfont\textbf{Parameterized Poker Squares}} % Article title

\author{
\large
\textsc{Danny Hanson and Chet Aldrich} \\
\normalsize Carleton College \\
\normalsize \href{mailto:aldrichc@carleton.edu}{aldrichc@carleton.edu},
\normalsize \href{mailto:hansond@carleton.edu}{hansond@carleton.edu}
\vspace{-5mm}
}
\date{}

%-----------------------------------------------------------------------------

\begin{document}

\maketitle % Insert title

\thispagestyle{fancy} % All pages have headers and footers

%-----------------------------------------------------------------------------
%	SUMMARY
%-----------------------------------------------------------------------------

\begin{abstract}

\noindent In order to create a solver for parametrized poker squares, we will attempt to use genetic programming to evolve an optimized strategy. Ideally, using parameterized input, we can develop a strong general solution that can be seeded to increase the effectiveness and speed of the algorithm's learning curve and produce high scores given minimal time for preprocessing. Then, in each game, we'll use a combination of a rules based approach early game, the genetic algorithm mid game, and a thorough search of the state space via an expectimax tree.

\end{abstract}

%-----------------------------------------------------------------------------
%	ARTICLE CONTENTS
%-----------------------------------------------------------------------------

\begin{multicols}{2} % Two-column layout throughout the main article text

\section{Introduction}

\lettrine[nindent=0em,lines=3]{T} he EAAI NSG Parameterized Poker Squares challenge takes a 5 $\times$ 5 grid where one can place cards, and with random draws from this deck, place cards such that the rows and columns of the grid make poker hands, with specific scores. The goal is to maximize the score based on cards you get, with one catch: you don't know the scoring system before you begin.

%------------------------------------------------

\section{Challenge Specifics}

The challenge starts by selecting a scoring system, after which our program will have 5 minutes to perform appropriate preprocessing. Then, for each game, our poker squares player will be given 30 seconds of total time for play decision-making. A player taking more than 30 seconds of total time for play decision-making or making an illegal play will score 10 times the minimum hand point score for the game.

Any number of scoring systems can be used, from the American and English scoring systems, which measure the difficulty of arriving at each hand in regular poker and specifically in poker solitaire respectively.
%------------------------------------------------

\section{Proposal}

We will attempt to use a genetic algorithm to optimize a rules-based approach. The concept is that we can abstract in-game partial hands (as part of a rules based approach), and then create a valuation function for each hand based on the parameterized input. From there, the abstractions can either be used or not used as part of the valuation function, and create many strings that denote using different game valuation abstractions randomly. By testing the fitness of each of the different strategies for valuation, the algorithm can evolve to use a more optimal strategy.

%------------------------------------------------

\section{Implementation: Training}


%------------------------------------------------

\section{Implementation: Early Game}
In the early game, there is little constraint that is added by the use of new cards. In particular, for the first five cards, placing the cards along the diagonal can prevent any of the first five cards that are drawn from being in scored hands with each other if they aren't deemed to be good cards to place together given the parameters.

Given our use of genetic programming, performing comprehensive searches of the state space is not the goal, so it may not present an issue to just use the developed tree during the training phase to make decisions.

However, another possible strategy for the early game is that we can create a rules based approach based on the intuition from having little constraint from placing the cards apart or together early game based on the `ideal' hands (those that award a lot of points).

We will attempt to build a rules based approach which determines the good hands for the particular game, and then uses rules that leave the possibility for the most of such hands available. Given it is effective in early trials, it will be added in the final version.
%------------------------------------------------

\section{Implementation: Mid game}

\subsection{Evolving an Optimal Strategy}
We intend to use a genetic programming based approach to create an optimal strategy for playing poker squares. We will be using a simplified functional language represented as a parse tree to decide where each card will be placed. This will involve the classic genetic programming steps of representing each chromosome as a tree, using each tree to simulate many (roughly 100) games, and then keeping the winners and performing both mutation and crossover randomly. The tree itself will take the current board and use various operation (such as addition and multiplication) as well as leaf nodes based on the state of the board (such as next card to place and random values) to deterine where the next card will be placed. The output of the tree will be a number between 1 and 25 representing the position on the board to place the card.

\subsection{Selecting Leaf Nodes}
The most difficult part of creating the tree will be deciding what leaf nodes to use and how to represent the board state. We would like to be able to use information such as where cards have already been placed, what the values of certain cards on the board are (both suit and rank), and what is the next card to be placed. This must be done in a way such that they will return only a single value that can be interpreted within the tree.

%------------------------------------------------

\section{Implementation: Endgame}
\subsection{Feasible Search}

In the late game (approximately the last four to five cards), the amount of possibilities in the state space greatly diminishes, which is a factor of more cards being drawn and fewer places to put the cards that are drawn. This makes it much more feasible to search the entire tree using an expectimax tree. Since the strategies used for the early and mid game are ideally going to choose quickly, we can exploit leftover time to search the state space an produce a more optimal outcome in the late game.

\subsection{Time Constraints}
At this point, it is unclear how much time is actually going to be available of the 30 seconds to search the state space before playing cards, and, more importantly, how feasible it is to search the state space late game. Ideally, we can create a function that checks the amount of time left and allots an amount of time to searching the state space before placing the cards.




%-----------------------------------------------------------------------------
%	REFERENCE LIST
%-----------------------------------------------------------------------------

% \begin{thebibliography}{99} % Bibliography: place potential sources here.

%Typical item shown below.
%\bibitem[Figueredo and Wolf, 2009]{Figueredo:2009dg}
%Figueredo, A.~J. and Wolf, P. S.~A. (2009).
%\newblock Assortative pairing and life history strategy - a cross-cultural
%  study.
%\newblock {\em Human Nature}, 20:317--330.

% \end{thebibliography}

%-----------------------------------------------------------------------------

\end{multicols}

\end{document}
